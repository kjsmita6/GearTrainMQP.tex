\section{User Study Survey}
\label{app:survey}
\subsection{Existing System}

\subsubsection{Survey Questions}

\noindent\textbf{Before the Task}
\begin{enumerate}
    \item Please enter your code (see the tutorial document for instructions on how to generate it). \emph{(Short answer)}
    \item What is your major(s)? \emph{(Short answer)}
    \item What is your minor(s), if any? \emph{(Short answer)}
    \item What year are you? \emph{(Multiple choice)}
    \item How familiar are you with gears (design)? \emph{(Ranking 1-5)}
    \item How familiar are you with gears (software use e.g. SolidWorks)? \emph{(Ranking 1-5)}
    \item Please give specific examples when you have designed gears in the past, if any. \emph{(Short answer)}
    \item How much experience do you have with SolidWorks? \emph{(Ranking 1-5)}
    \item Please give specific examples when you have used SolidWorks in the past, if any. \emph{(Short answer)}
    \item Have you used the gearbox design software before? \emph{(Yes or nN)}
    \item If you responded yes to the previous question, do you have any comments about it, such as feedback, suggestions, or problems? \emph{(Short answer)}
\end{enumerate}

\noindent\textbf{After the Task}
\begin{enumerate}
    \item Please enter your code (same as the pre-survey, see the tutorial document for instructions on how to generate it). \emph{(Short answer)}
    \item Were you able to complete the tutorial? \emph{(Yes or no)}
    \item Please upload a screenshot of your completed design in SolidWorks. \emph{(File upload)}
    \item Approximately how long did it take you to complete the tutorial (mm:ss)? \emph{(Short answer)}
    \item How difficult was it to complete the tutorial? \emph{(Ranking 1-5)}
    \item How difficult was it to modify the gear data? \emph{(Ranking 1-5)}
    \item What about the gear design software did you find easy, if anything? \emph{(Short answer)}
    \item What about the gear design software did you find difficult, if anything? \emph{(Short answer)}
    \item What would you change in the current system to make it easier to use or more functional? \emph{(Short answer)}
    \item What are your top 3 or favorite features? \emph{(Short answer)}
    \item Any additional comments or other feedback that you would like to give? \emph{(Short answer)}
\end{enumerate}

\subsubsection{Open Response Question Results}

\noindent\textbf{What about the gear design software did you find easy, if anything?}
\begin{itemize}
    \item I found the whole software pretty user friendly, I didn't have any problems following the tutorial and I was able to successfully generate the gears.
    \item The layout was intuitive and not a lot of instruction was needed. The labels of the data input boxes were clear and explanatory
    \item It makes it easy and efficient to generate a gear box and correct dimensions with no errors, it is very easy to use and fast.
    \item I really like how simple and easy to use the GUI is, I also like how there was no real way to mess it up
    \item Designing the gearbox was easy. It was very clear where all the numbers needed to be plugged in.
    \item software was easy to use, I liked how it generated all the parts and assembly for you quickly
    \item it was easy to input the data and convenient that there was a button to import to solidworks
    \item you don't have to actually design the gears and shafts they just created them for us
    \item I thought it was pretty easy to follow, very straight forward and user friendly.
    \item You just had to put in the information, you did not have to do any calculations.
    \item it created gear sets much faster than creating them in Solidworks alone.
    \item The procedure was pretty straightforward, the data was easy to change
    \item The UI screen to input all the data given was easy to fill in
    \item The user interface was very user friendly and easy to use.
    \item The way the software compiled the gears into solid works
    \item The inputs for the numbers were very straightforward.
    \item The layout was very intuitive and simple to access.
    \item The imputs and it generated the gears pretty well
    \item it was easy to put in the gear information data
    \item typing it into the boxes seemed simple enough
    \item it was just inputing data given, fairly easy
    \item To download and have it generate seemed easy
    \item Watching them gears mate themselves.
    \item it is easy to add a new set of gears
    \item user input was very straight forward
    \item The initial UI was very simple
    \item everything was organized well
    \item Everything is clearly labeled
    \item when modifying the gear data
    \item Easy understandable layout
    \item it was all very easy to do
    \item Almost everything is easy
    \item Easy to put in values
    \item entering in the data
    \item putting indata
    \item yes very easy
    \item clear labels
\end{itemize}

\noindent\textbf{What about the gear design software did you find difficult, if anything?}
\begin{itemize}
    \item I was a little confused during the file saving part of the program, I wasn't sure if I was making a file for my gear assembly or if I was trying to find the location of solid works. Maybe I just wasn't paying enough attention though.
    \item If I was using this software on my own, I might have difficulty visualizing the placement of the gears. However, the auto-mesh feature seemed to account for this somewhat.
    \item After a second gear set is added, the page won't scroll down to completely reveal the ``Click to Add Gear Set'' button. Only the top of it is visible.
    \item I just found it a little bit tedious to input all of the information but I know it is necessary to model the gear the right way
    \item nothing but it did freeze my computer and I had to close solidworks and try again but i think it was just a solidworks error
    \item Downloading the file was a little difficult because the computer didnt download it right away. There was an error to change.
    \item I wish that you didn't have to put as much information or that it was more user friendly because it seemed a little tedious.
    \item Exporting it was a little difficult. I did miss a step and don't have a shaft crank because I got a little confused.
    \item the green box to add a gear set wouldn't show, could not read no matter how far I tried to expand the window
    \item The pictures don't entirely match the table of data so it threw me off for a second before I realized it.
    \item Adding a second gear set was tricky, but mostly due to a lack of my own understanding.
    \item some numbers in the table weren't the same as what was shown in the final screenshots
    \item I'd have to play around with it more to answer this question accurately.
    \item when to open the generated gear with solidworks it was a bit confusing
    \item Probably just knowing what to do/getting my antivirus to trust it
    \item the last few steps were confusing because I didn't get a pop-up
    \item I had to select the location of the executable for solid works
    \item the ratio picture in the beginning of the packet threw me off
    \item I dont know exactly how I would choose some of the objects
    \item Difficult to know what dimensions to put in for the gears
    \item not sure how to get this data or what it means
    \item getting it to transfer over to solidworks
    \item not knowing enough about gears
\end{itemize}

\newpage
\noindent\textbf{What would you change in the current system to make it easier to use or more functional?}
\begin{itemize}
    \item I don't know what the intention of the use of the program is but as someone who knows nothing about gears I don't know what some of the things mean. Ex. I don't know what the different types of gears look like or what an idler gear is.
    \item The only problem I noticed was the add gear set button moved off the screen after the second set was added, there was a scroll bar but it did not go down far enough
    \item Instead of telling people to wait for the ``Press OK once Solidworks has loaded'' popup, maybe just indicate that this is only needed if SW was no already opened
    \item The window of the program didn't adapt to that of my screen, so the majority of the ``add new gear set'' button was hidden after adding the second gear set
    \item I am fairly unexperienced when it comes to solid-works so I had no idea what I was building until the program built it for me, so maybe a live image.
    \item I would remove the grey shading because it seems as though you cannot change the numbers even though that option is available.
    \item The way it generates in solidworks has an extra step I'd try to get rid of (when you have to click more info \& run anyway)
    \item A hover feature, where if you hover over the fields to be filled out a description or animation is able to be viewed.
    \item I think the system is pretty good just takes a while but its still better than manually creating all of these parts
    \item my program was a little slow to open all the solidworks parts and the assembly. Thats probably just my computer
    \item Maybe have the option to make individual gears instead of them in sets so you could have an odd number of gears
    \item Explain what some of the different parameters are. Lots of people don't know terms like ``pitch angle''.
    \item It would be cool if it automatically made a folder that held all the files it's creating.
    \item add a help feature to explain some of the parameters (mainly for newbies like myself)
    \item allow more than 2 gears to be on a shaft(not sure if program can already do this)
    \item finding an easier way to transfer over to solidworks, My solidworks did not open
    \item Giving a preview would be awesome, but it could also take up a bunch of memory
    \item open solidworks automatically? Or a note on the side says pre-open solidworks
    \item Maybe more visual in the application before solidworks, preview.
    \item If possible, maybe include some sort of preview of the gears.
    \item Unsure, maybe just make sure the pictures match for beginners
    \item I would add a more simple way to visualize the end result.
    \item Potentially have a preview of the locations of the gears
    \item allow for a faster way to generate the gear assembly
    \item This is a bit out there, but possibly a visualizer?
    \item better GUI, less steps to do a simple generation
    \item instruction to find the solidworks.exe file
    \item Improve the layout of the UI for clarity
    \item give more info about what is happening
    \item make sure the pop-ups pop up
    \item Maybe just UI improvements
    \item the software looks dated
    \item easier to save
\end{itemize}

\subsection{New System (GearTrain)}

\subsubsection{C21 Study}
\paragraph{Survey Questions}
\label{app:survey_c21_ques}

\noindent\textbf{Before the Task}
\begin{enumerate}
    \item Please enter your personal code. The code is generated as follows:
    \item What is your major(s)?
    \item What year are you?
    \item How familiar are you with gears (design principles, terminology, etc.)?
    \item How familiar are you with CAD (e.g., SolidWorks, AutoCAD)?
    \item Have you ever used a gear design software before (e.g., geargenerator.com)?
    \item If you answered ``Yes'' to the previous question, do you have any comments about it or things you would change?
\end{enumerate}

\noindent\textbf{After the Task}
\begin{enumerate}
    \item Please enter your code (same as the pre-survey, see the tutorial document for instructions on how to generate it).
    \item Were you able to successfully complete the tutorial?
    \item Approximately how long did it take you to complete the tutorial?
    \item Please upload a screenshot of your resulting design from SolidWorks (guided design).
    \item Please upload a screenshot of your resulting design from SolidWorks (unguided design).
    \item How difficult was it to complete the tutorial?
    \item How difficult was it to modify the gear data?
    \item How difficult was is to navigate through the program?
    \item How many times did you have to consult the help website?
    \item How difficult was it to complete the unguided section, given all that you had learned in the guided section?
    \item How satisfied are you with the look and feel of the program?
    \item What about the software did you find easy?
    \item What about the software did you find difficult?
    \item If you could change/add/remove anything in the software, what would it be?
    \item An old version of this software did not have the 3D viewer and it was all text based. Did you find that the 3D viewer improved your experience with the software?
    \item What is your favorite part about the software?
\end{enumerate}

\paragraph{Open Response Question Results}
\noindent\textbf{What about the software did you find easy?}
\begin{itemize}
    \item It was very clean and clearly labeled, easy-to-follow what inputs were being added.
    \item I really love the auto-mesh feature, it seems like it will save people a lot of headaches!
    \item Very easy to modify gears and their properties
    \item It was really nice that the gears were auto meshed and that the CAD files were developed for you. I spent way to much time in RBE2001 designing teeth on gears in solidworks.
    \item Changing the gear parameters was very easy and having the upper bar with some of the major operations was very nice
    \item Inputting data was very easy
    \item easy to edit gear data, ui is nice and simple, controls were intuitive 
    \item Data entry was very simple, creating the model in SolidWorks was also very easy
    \item Entering data and converting to SolidWorks were both very easy
    \item The auto mesh feature seems very useful.
    \item The software installed very easy, had a great and simple UI, was very stable, and transferred data over to SolidWorks without any issues. No navigation issues in the software, and the UI looks pretty solid and modern for being under-development.
    \item The gear editing tools and buttons were very user friendly and self explanatory for the most part.
    \item Generally very quick to learn the basics, minimal unnecessary complication
    \item I liked the auto meshing and ability to generate it in solid works.
    \item It was very easy to navigate and modify the gears.
    \item changing all the parameters was very straightforward and uploaded into solidworks well
    \item It was super easy to navigate and understand the controls
    \item It was very easy to generate the gears sets with no issue.
    \item The quick edit and add features
    \item Overall it was quite easy to use
    \item The buttons were large and easy to find.
    \item The software was able to create properly meshing gears with minimal input and little math on my part.
    \item Was straight forward and simple. Was cool to watch the parts being made in solidworks 
    \item Simple button layout
    \item I loved how easy it was to create a gear box and have it just create a Solidworks file for me. It makes it so much easier and saves a ton of time. 
\end{itemize}

\noindent\textbf{What about the software did you find difficult?}
\begin{itemize}
    \item For the unguided section, I had to close both out of Solidworks and the gearbox software completely, in order to generate my modified unguided file (I had only closed out of Solidworks after having finished the first section).
    \item Having to back out of the main menu to add another gear set. It wasn't difficult, but if you have a system with a ton of gear sets I could see it being a little annoying. It's really not a big deal in the long run, but having a button to add a new gear set on the top ribbon could make it just a little bit quicker to use
    \item Was hard to determine when Solidworks was done or if I needed to be doing anything while it was generating the Solidworks model. If there's some way to have Solidworks do the work in the background without showing the user anything until the final design, that would be less confusing. This way, the user would stay in the GearTrain software (probably looking at a progress bar) until Solidworks is completely done creating the model.
    \item I was a bit confused by the ``back'' and ``revert'' buttons, I expected there to be a ``save'' button so I spent some time looking for that before discovering that clicking a new gear to edit did not remove the changes to the previous gear.
    \item Some of the navigation from gear to gear was a bit difficult but not bad
    \item I didn't find anything difficult
    \item my middle mouse button doesn't function as a button, only as a scroll wheel, so being able to rotate and pan like in solidworks with right mouse button would be nice, or at least have the configuration option to do so. also, the arrow keys to move at the bottom are backwards?
    \item I did not find anything in the software that felt difficult to use
    \item I did not find anything particularly difficult
    \item The need to close all SolidWorks tasks is a bit annoying.
    \item The UI had a few scaling issues with a smaller laptop screen (the buttons along the bottom of the screen were covered up by the windows start bar, and I had trouble finding the ``back'' button for a little bit). Additionally, the input boxes to modify gear parameters sometimes didn't select the whole value when clicked, which was slightly annoying but only delayed my progress by a miniscule amount.
    \item Some of the technicalities, like making sure solidworks was completely closed and not pressing ok until the program had fully loaded.
    \item While the visual is useful, its hard to get a sense of scale from it
    \item The difficult part for me is getting the coordinates right 
    \item Nothing.
    \item I was looking for a save button but I like that it autosaves 
    \item Sometimes the transition from the program to solidworks was a bit finicky at times 
    \item Inputting numbers on the x, y, and z plane were sometimes annoying to do. The zeros should disappear when clicked on.
    \item The inability to name the gears
    \item Nothing much, but loading into SolidWorks would sometimes be a pain
    \item The software was new, so it took time to familiarize myself with the software and remember all required steps to get through the unguided portion of the tutorial.
    \item I would have liked it if the back button after editing a gear shaft was a save button. As is I wasn't sure upon clicking it if my work was going to be saved or not. Obviously the save box that comes up answers that but it would still be nice to have a save button
    \item Nothing really
    \item None
    \item Sometimes its hard to tell if you clicked on something as it doesn't highlight or it would be hard to know if something was saved after you changed the properties of a gear set because there was no save button. 
\end{itemize}

\noindent\textbf{If you could change/add/remove anything in the software, what would it be?}
\begin{itemize}
    \item Maybe a minor change - a highlight for when selecting the gear set that is about to be edited. I figured it out right away, but at first clicked straight to ``Edit'' instead of selecting the ``Gear Set 1'' and then ``Edit,'' since that was the only set in the menu when starting.
    \item As I said above, I would have Solidworks work in the background while the user looks at a progress bar in the GearTrain software. This way, the user would know that Solidworks was generating the model, but they wouldn't have to see a bunch of different screens pop up and disappear. This would also reduce the chance of error as the user would not have the opportunity to click something in Solidworks and disturb the software until the model is complete.
    \item Perhaps a way to save changes to the gears rather than saving upon hitting back, I was looking for a way to apply changes. It might also be helpful to be able to see the names of gears within each set from the set list.
    \item fix the arrow keys, add the feature to snap to a view like in solidworks (left right top bottom front back etc)
    \item I would allow for an easier and faster was to switch between editing gear sets, such as a tool on the header of the software window
    \item I would allow for an easier way to navigate between editing different gear sets
    \item I think the software is brilliant and I would love to see it be implemented in a feature-based cad program! If anything, having the program embedded into SolidWorks would definitely be a long-term goal, but the standalone UI was simple enough and very practical to transfer data over. Feature-wise, I would love to see more adjustments regarding shaft and bearing info, and maybe some parameters to adjust shaft and gear meshing tolerances, but most of these things are a convenience factor because the parts are added into SolidWorks as editable features and can be later modified by the user.
    \item Add a button that saves your gearset updates so there isn't a popup every time you press back to update a new gearset.
    \item The minimum window size is annoying, as is the fact Solidworks has to be closed to generate the gear assembly. It would probably fit a realistic workflow better if it the window could be made arbitrarily small (to allow for other programs to be visible at the same time), and if the program was able to generate the Solidworks files without having to close out of Solidworks beforehand
    \item I think this is very good, but maybe the ability to add a motor and it generates the torque and gear ratio
    \item More Color
    \item I'd like to name the gears
    \item Easy access to gears from other sets, auto save changes to gears when switching between
    \item A save button after you change the gear properties. 
\end{itemize}

\noindent\textbf{What is your favorite part about the software?}
\begin{itemize}
    \item It was a straightforward, simple input into the software that (magically) translated over to a full Solidworks assembly file!
    \item The Auto Mesh feature
    \item I like the AutoMesh feature because it allows you to be imperfect in your placement of the gears.
    \item The fact that it generates the solidworks files, doing gears by hand in solidworks is a hassle.
    \item It was overall very easy to edit the gears themselves once you get to the gear and the Solidworks generation is fantastic
    \item I appreciate the user friendly design of the software
    \item easy to use, simple, straightforward, the 3d viewer is also nice, can't imagine it without it
    \item The ability to convert the created gear boxes into SolidWorks models
    \item The ability to convert the gear boxes into SolidWorks
    \item The user interface is intuitive and looks pleasing.
    \item Clicking ``Generate in SolidWorks'' and watching it rapidly build an assembly that would have taken me quite some time to design from scratch. I was amazed! Also, I really liked how it built the parts internally using SolidWorks, because dealing with .step or .stl files is often very annoying. This program offers a great way to speed up the process of gear-train design without restricting the designer from adjusting what the program has made later on.
    \item How easily it transfers the numerical data into a fully functioning solidworks assembly.
    \item The auto mesh feature is simple to use and very convenient
    \item The ability to visualize your gear before sending it to SolidWorks
    \item The 3D viewer
    \item simple to use and updates the visual as I plug in values
    \item The simplicity to create complex trains
    \item Like mentioned above, but the 3D aspect of the gears was awesome.
    \item Soldiworks assembly generation
    \item It was easy to use overall
    \item I like that the 3D viewer updated as you changed the numerical values.
    \item The auto-meshing function was very neat to see.
    \item Watching it upload into solidworks 
    \item Auto generation in Solidworks
    \item That I get a 3D visualization. 
\end{itemize}

\noindent\textbf{Are there any additional comments or other feedback that you would like to give?}
\begin{itemize}
    \item This seems like a promising and incredible software development! Nice work, and I look forward to seeing its completion!
    \item your unguided tutorial coordinates are wrong for gear d, not sure if this was intentional or not but i just did it according to the coordinates that were given
    \item The bounding box generated had an over defined sketch, so I am not sure what caused that.
    \item Great program that I am looking forward to see the bright future of!
    \item Just amazing job, this is really cool, i'm really impressed 
    \item The second link to the unguided section would not open for me, so i just used the table to generate the new gear train
    \item I was unable to generate a model in solidworks. I made sure there were no solidworks programs running in the background before attempting but when I chose to generate the model in solidworks it was able to open solidworks but didn't do anything further. This was using the elabs.wpi.edu remote desktop
\end{itemize}

\subsubsection{D21 Study}
Note that the D21 and C21 study use the same survey questions, so see Appendix~\ref{app:survey_c21_ques} for the questions used in both studies.

\paragraph{Open Response Question Results}

\noindent\textbf{What about the software did you find easy?}
\begin{itemize}
    \item The interface was simple and all the necessary functions, menus, and options were easy to find.
\item It was very easy to change values, it felt very user-friendly in that regard
\item Nothing was difficult to find, text boxes and all input fields were very clearly labeled.
\item I found it easy to perform operations within the program, everything was clearly labeled and separated into organized groups.
\item Shifting between each gear set and inputting gear dimensions was super easy and helpful.
\item I liked that there were preset parameter which could be changed if needed. Definitely made things easier since there were no errors. 
\item The user interface
\item I thought finding and navigating the different features was very easy. All of the gear options were clear and easy to see.
\item It was an easy interface to work on with very little issue navigatig
\item Everything i needed was right on the screen i was in
\item It was easy to change the numbers for gears.
\item its an easy ``plug and play''
\item Editing the gear data and adding additional gear sets
\item Basically did everything for me. I had to do little effort 
\item Literally magic its so easy I like this a lot
\item auto mesh
\item There were few buttons, which made it easy to know what each one did
\item Selecting the individual gears on each shaft, as well as modifying the information for each gear.
\item I found it easy to change different filters of lengths, number of gear tooth, and other filters in the software. The buttons were all self explanatory. 
\item The layout was very simple and easy to navigate
\item It is very simple to edit the data
\item it was very user friendly and automatically did many of the difficult tasks such as automeshing
\item It was pretty straightforward
\item It was extremely easy to edit the gear sizes, add gears, etc. 
\item moving the gears about the cooridnates
\item It was easy to edit the data and add new gear sets.
\item The UI
\item Editing the gears and being able to auto-mesh the gears
\item It was easy to find things
\item I found the automesh feature really easy and it was very clear what gear you were editing.
\item Adding and modifying the shafts and gears
\item Doesn't show me any options or information that is overwhelming. It's very simple
\item Very User friendly, clearly labled.
\item worked well, easy to use generally
\item The interface was very simple and easy to understand
\item Inputting different paramaters
\item It was nice to be able to edit the gear properties and directly export the information in the app to the SolidWorks app
\item I thought it was very easy to modify the gears and add now shafts.
\item The auto mesh is a good feature
\item the easy editing of the parameters and generated model in solidworks 
\item the way to access the other gears and their settings
\item making the changes
\item The auto-mesh feature was very nice to use. Made the process very simple and easy to follow
\item It was easy to modify the gear data and generate the gears on solidworks
\item Modifying the values
\item This software was simple and very easy to navigate.
\item the software was pretty intuitive if you had the guide. 
\item Input of the gear parameters ran very smoothly
\item It was easy to go in and modify the dimensions of the gears
\item modify data section is pretty good
\item The way it was mapped out in the help document. It was very easy to follow step by step.
\item Entering the data
\item Yes it was very satisfying.
\item UI is intuitive
\item It was easy to automesh the gears and upload it to SolidWorks.
\item The ui was functional
\item I think the software was pretty easy to use
\item It was very easy to input the data and have the program auto mesh the gears for me. 
\item It was easy to auto mesh and generate in solidworks
\item The way it transfers to solidworks and the user-friendliness, I have absolutely zero experience with any of this and it was very easy to follow. also the fact that you don't have to constantly worry about saving, you always know if you're saving or not saving. 
\end{itemize}

\noindent\textbf{What about the software did you find difficult?}
\begin{itemize}
\item The gear options were a bit difficult to differentiate because they are all close together.
\item The rotate feature to see if my gear configuration exactly matched the expected result was a little difficult to use and did not feel as fluid as SolidWorks
\item I didn't like how ``gear sets'' were a separate selection from the individual gears, a setup like this wouldn't allow a user to have two gears driving a single gear (as is common in many FRC gearboxes for example).
\item \& of teeth shouldn't be a dropdown, that was a little hard to use.
\item I know they weren't part of the study, but I couldn't seem to figure out the ``Analysis'' boxes (possibly unimplemented as of yet)
\item I don't like the way you have to input x, y, and z. Gear 1 should exist at 0,0,0 and the meshing gear should have an angle relative to it from the horizontal axis. For gears on the same shaft, I just the spacing between them. This way x, y, and z are autocalculated based on angle and gear diameters. This avoids the problem of me needing to do trig to figure out where a gear should go, or the automesh moving my gears to places where I do not want them. You'll notice my unguided gearbox is rather different to the one in the tutorial despite the fact I entered the provided values.
\item I struggled with the unguided section, I was easily able to input the data and automesh the gears in the gear train program. It was difficult to send my stuff over to solid works without issue. my guided section worked but not my unguided. I was experiencing computer hardware issues seemingly, as my screen turned totally gray.
\item I had some trouble transferring the data into Solidworks. I could not find the file to use as a location for the data. I did find it eventually, but that was the only part that took a little longer to complete.
\item It was time consuming to highlight each parameter that had to be changed in order to enter a new number.
\item I don't think there was any difficult part of the software. I followed the tutorial and didn't have any major issues. I did forget to shut down solidworks entirely and something was still running, which made the second generation not work, instead, what looked t=like a single gear part file was loaded. After making sure everything was shut down in task manager and generating again, it worked fine. I did it through the remote desktop and the longest part was waiting for the program to download, extract, and run for the first time.
\item There was no save button.
\item loading time
\item When generating in SOLIDWORKS, the computer did not know which program it needed to open, so it tool me a while to locate solidworks for the program. But this issue could've been more user error than the program itself having an issue.
\item wasn't clear at first how to edit things
\item It was a little confusing at first about how the gears were just generated
\item It was confusing exporting to Solidworks and having to wait for it load before continuing. 
\item Putting in the numbers was a bit annoying when they were already filled out
\item at first, I was confused on how to add the input shaft because I thought it would just generate that (not the gears and the shaft)
\item It was unclear how to delete an accidentally added gear, and the performance on the preview was bad so it was hard to preview in the gear program before generating in solidworks
\item It was difficult for me to export the second model to SolidWorks correctly. The model in Gear Box looked much different than the model that was generated in solidworks. 
\item The gears look like the bearings, so you can click on the gears but not the bearings. The color choice is also not the best, a higher contrast environment, and a color difference between the gears and bearings would be nice
\item I could not get the solidworks file to generate. I followed the tutorial steps as well as the video steps, but it never loaded. It also never saved anything to the automated design/gear design output directory, even when I had all file types listed. Essentially, I made the design, clicked generate in solidworks, checked the box to add the crank shaft, waited to click ``ok'' until solidworks had fully loaded, but nothing happened and I couldn't load the files from the folder as nothing was there.
\item Locating Solidworks on my computer
\item Uploading to Solidworks, it took me like over half an hour both times trying to get the files to actually load in Solidworks
\item I wish that when you click to edit a number it automatically highlights it all to make it easier.
\item Understanding which gear shafts it would move in order to automesh the gears.
\item Figuring out the gear ratios
\item Cannot change all gears within the same screen
\item Not sure if this is on your end or wpi remote desktop, but solidworks was not working
\item glitch where i didnt have anything opened and it says 'save changes to existing project' when i try opening something. very finnicky, have to restart to generate to solidworks
\item It was an inconvenience to make sure all of solidworks was closed in order to generate the two configurations in solidworks.
\item It did not generate the gears in SolidWorks, despite opening SolidWorks. The pictures are from the tutorial, just to fill out the required field.
\item I could not make the unguided test portion work. The file opened in the GearTrain App and I was able to add the second gear but the model could not export to SolidWorks no matter how many times I tried or waited. I am unsure if it is an internal error or some app difficulties. I have posted the completed design in the GearTrain app as a submission.
\item Going from one set to another felt a little clunky
\item understanding the additional info required to generate 
\item that i can't make the screen small so i have to open and close the tab
\item I was confused how the program chose which gears to auto-mesh together. When I was making the gears, it was hard to tell which was supposed to mesh with which others
\item I needed to download solidworks because the remote desktop was sometimes unresponsive after generate the gears in solidworks. Once it was downloaded I found it easy to modify the gears
\item exporting to solidworks but it was my own fault for not waiting until it fully opened. 
\item I just had trouble opening the unguided file at first, but I think that was more of a Windows issue than a software issue.
\item maybe if there was a save changes button instead of clicking back to save 
\item Opening a the half way completed file over remote desktop was a little tricky. It wouldn't allow me to open the file type with the gear software.
\item It didn't seem like the software flowed well, drawings could be clearer
\item I have little experience with SolidWorks so navigating the software will propose some difficulties down the road. 
\item I found rotating the image to be difficult.
\item I followed the instructions, even confirming that i had in fact done the correct thing with james scherik, and it still didn't generate coherent solidworks files
\item I found some of my values seemed to change after editing them when I would go back to view it a second time and I'm not sure why 
\item kind of insignificant, but I had trouble figuring out how to add the second gear set for gears C and D based on the tutorial, but even then it only took a couple minutes to play around with.
\end{itemize}

\noindent\textbf{If you could change/add/remove anything in the software, what would it be?}
\begin{itemize}
    \item Space out the gear options slightly.
\item I work work on the rotation feature and make it smoother and easier to turn. Also maybe add an ``add gear assembly'' when working in another instead of having to go back out to the home window and manually add another assembly.
\item Get rid of the concept of gear sets entirely, there should just be a single list of gears. Then through either checkboxes or an intuitive UI solution (like the Solidworks design tree) a user can select which gears mesh with each other and which gears sit on the same axles.
\item Please allow for multiple gears to drive one gear.
\item As I mentioned in the ``What about the software did you find difficult?'' question, please change the way you set the gear position.
\item Proper trackpad support for the viewer. (right click to pan instead of middle)
\item Would love a feature where I could import a dxf or a 2d drawing of my gearbox. Back in FRC (sorry to keep bringing it up, just have a lot of experience with gearboxes from First) designing the gearbox was the easy part: You know where you want your motors, where you want your output shaft, what your desired drive ratio is, and what gears VEX.com sells. From there it's just a puzzle of connecting things. What would be awesome is using this software as a tool to not only generate the internals (so I don't need to hand place bearings, shafts, lock collars, and gears) but also to tool to do analyses on the gear train (to see if I might need to swap Aluminum gears for steel or change intermediate ratios or make any other changes to the gearbox).
\item In the software, I would change the communication with solidworks, and perhaps try to use less processing power as solidworks already uses a lot of processing power on my computer.
\item I'm not too familiar with gears, but I'm wondering if there is a reason why all the separate parts must pop up before the assembly is created. While it's nice to see all the parts, maybe it would be less cluttered if only the final assembly pops up in SOLIDWORKS. 
\item Lots of the editing options in the gear menu were done by dropdown, for instance the shaft orientation and the number of teeth. This worked fine, but it could be nice to have the option to be able to type them in or use pulldown.
\item Add a save button.
\item Being able to automatically open solidworks without me needing to tell the software where it is located
\item I cannot think of anything 
\item Can the gearbox generation software wait for Solidworks to finish loading in the background so that it can seemlessly generate the gear train?  And also can the generation software kill the Solidworks processes if they are already running?
\item I would edit the feature for transporting the file to solidworks and possibly make it easier or be able to open the solidworks file from solidworks directly. 
\item I would make it possible to edit gearsets by double clicking on the gearset
\item I would not change anything, extremely easy to use. 
\item I would make it so that you can just click on a gear to select it, no matter what gear set you are in. 
\item I would add an option to save your file. The first time I had to create a new gear set, I wasn't sure if clicking the back button would delete the work I had currently done. Alternatively, you could add an option to add another gear set directly on the page where you're editing the currently selected gear set or an option to move between already created gear sets.
\item Having to locate solidoworks
\item Having a undo and redo button and I would add a preview if possible of what it would look like in solidworks before you generate it.
\item Little tabs which give information about each property. ie: what different pressure angles and pitches will effect
\item Just make it easier to generate the configurations into solidworks
\item I feel the software is great and easy to use but for some reason exporting some files seem to stump SolidWorks from extracting them. 
\item There isn't anything that I can think of that I would like to change. 
\item When trying to edit the systems, everytime you want to switch betwee sets you have to answer a popup, I feel like the pop up shouls only be there if you were going to delete and not just change onne variable.
\item being able to make the screen smaller 
\item I think options for planetary gears/inverted gears would be cool. 
\item I liked the software and found it simple to follow
\item Maybe just enhancing the visuals a bit
\item As a beginning user of the software, there is nothing I would change about it.
\item Edit ``back'' to something like ``save'' in the editing menu. It took me a little to realize that back would save my work and get me back to the main menu to add another gear stage.
\item i may want to add teeth into the pictures
\item I wouldn't change anything
\item Autofilling the information from previous gears would be nice.
\item I has the gears in their correct positions and everything in Geartrain, and then they generated very differently in solidworks, the second time they didn't generate at all
\item The 3d viewer seemed inaccurate for the last gearset, it did not look the same in solidworks. 
\end{itemize}

\noindent\textbf{What is your favorite part about the software?}
\begin{itemize}
    \item It's very clean
\item How easy it was to edit the properties of the gears and add new gears.
\item It was super easy to use and it took me no time at all for any needed feature of the gear assembly
\item Exporting to SW in a single click, no gear generator I've used packages gearboxes so neatly!  
\item my favorite part about the software was the inviting user interface, it was fast to open up and it was quick and efficient until it had to communicate to solidworks
\item I enjoyed the auto mesh gears button which allowed the entire gear configuration to come together and work as it should.
\item It looks very fun and useful! I enjoyed seeing all the parts come together in SOLIDWORKS. 
\item Auto Mesh Gears
\item I was surprised by how easy this was to use. It worked pretty conveniently and the generation into SolidWorks was smooth and cool. I'm not super familiar with other gear generation options, but I could see myself using this.
\item I like watching the gears change as I changed data, it was good to have the visual to check in case I mistyped.
\item how easy it is generated in solidworks
\item The 3D viewing element was beneficial for 3D visualization.
\item Easy to use and convert to solidworks
\item Once I got it all set up, I found the software very easy to navigate and very intuitive! I was able to complete the unguided design task in like 2 minutes. So once I got the hang of the software, it was very quick and easy to use.
\item The 3D viewer and the fact it builds everything in solid works for me 
\item Huge process made in clicks, cad magic is awesome
\item how easy it made the assembly in solidworks
\item It was really easy to convert the model to a Solidworks file
\item Seeing the transformation of the gear train from the generation software to Solidworks.
\item My favorite part of the software is that is self explanatory and any one would be able to use it. 
\item The name is nice
\item The auto mesh is a great feature but then it disregards multiple coordinates originally inputted into the table.
\item very easy to edit properties
\item The Solidworks generation is really awesome, especially how it automatically mates everything in addition to the parts.
\item How models can be exported to solidworks. 
\item The auto mesh gears was so satisfying, best part.
\item I would have really liked the option to export directly to solidworks, had I been able to get that to work.
\item simplicity
\item Editing the gears and and being able to auto-mesh the gears
\item Visual elements
\item That it automatically creates a working CAD assembly as well as all the parts
\item The transformation of it to solidworks
\item Instant 3D modeling 
\item Very easy to navigate
\item generally easy to use and helpful
\item It was very easy to understand and learn how to use
\item GUI Ease
\item How easy it is to manipulate and change the gear properties 
\item I liked how easily it was able to generate the geartrain in solidworks.
\item making gearboxes manually sucks so this is very nice
\item this is going to make many projects much easier, than to make or find the right parts and assemble them 
\item Very fast and easy to use I liked the auto alignment
\item it is really easy to use and make the 
\item Was super easy to use. No bugs or major confusion
\item My favorite part was the 3D viewer because you can easily modify the gears and see the design in the viewer and easily generate it in solidworks.
\item The ease of use
\item This software was engaging and quite interesting. I had a lot of fun even though it was very straightforward. I can ultimately credit this enjoyment to the SolidWorks simulation after I edited the gear properties.
\item it is simple to use and integrated really well with solidworks 
\item The visualization made it very easy to see any input errors. 
\item All dimensions of the gear are in one place, easy to change desired data
\item the 3d view
\item The visuals it gives. It makes it easier to imagine the model.
\item Definitely the 3D viewer
\item The easy way of changing the parameters of the gears and the auto mesh feature.
\item auto-mesh feature, 3D visualization
\item Probably having a 3d visual.
\item the 3d viewer
\item It was very interesting to see the gears I made be generated
\item the 3d viewer really helped me get an idea of what I would get. 
\item I definitely liked the 3D viewer because it is hard to visualize as a beginner
\item how straightforward it is! If I want to make a gearbox, simple or complicated, beginner level or not, I know exactly how
\end{itemize}
